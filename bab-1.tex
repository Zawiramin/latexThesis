\chapter{Pendahuluan}

\section{Pengenalan}
Bab ini menghuraikan pengenalan kepada kajian secara komprehensif. Perbincangan meliputi latar belakang kajian yang menghuraikan isu data raya dalam konteks pendidikan, peramalan pencapaian akademik, dan keutamaan penggunaan faktor non-kognitif dalam model peramalan terhadap murid yang menduduki ujian pentaksiran antarabangsa, PISA. Pernyataan masalah pula merangkumi permasalahan dalam membina sebuah model peramalan pencapaian pendidikan murid dan faktor non-kognitif sebagai pemboleh ubah utama dalam pembinaan model yang dipercayai antara faktor terpenting lain yang perlu diberi perhatian dalam penentuan pencapaian dan hala tuju masa hadapan murid. Bab ini juga menggariskan tujuan, objektif, persoalan, dan hipotesis kajian serta kepentingan dan batasan kajian ini juga turut diperincikan. 

\section{Latar Belakang Kajian}

\section{Pernyataan Masalah}
\subsection{Hipotesis Kajian}
\subsection{Persoalan Kajian}
Berikut adalah persoalan-persoalan kajian yang akan diterokai:
\begin{enumerate}
  \item Apakah wawasan yang diperolehi melalui teknik perlombongan data pembelajaran berpenyelia terhadap data PISA Malaysia?
  \item Sejauh manakah penggunaan model peramalan (pembelajaran berpenyelia) dalam data pendidikan peringkat sekolah?
  \item Apakah kesesuaian penggunaan teknik pembelajaran mesin pembelajaran berpenyelia dalam membina model peramalan prestasi murid di Malaysia?
  \item Apakah kelebihan penggunaan teknik pembelajaran mesin (pembelajaran berpenyelia) ke atas teknik-teknik statistik tradisional dalam meramalkan pencapaian murid sekolah?
  \item Apakah cara untuk merangka model pepohon keputusan (pembelajaran berpenyelia) bagi sebuah model peramalan prestasi murid di Malaysia?
  \item Adakah model lain pembelajaran berpenyelia mampu memperbaiki ketepatan model peramalan pepohon keputusan?
  \item Domain non-kognitif kurang mendapat perhatian. Apakah dapatan ramalan menggunakan teknik pembelajaran mesin pembelajaran berpenyelia terhadap faktor non-kognitif sains pelajar dalam ujian pentaksiran PISA 2015?
\end{enumerate}

\section{Tujuan Kajian}

\section{Objektif Kajian}

\section{Skop Kajian}

\section{Metodologi Kajian}

\section{Sumbangan Kajian}

\section{Organisasi Tesis}

\begin{table}[hbt!]\centering
\caption{Bilangan permata}
\begin{tabular}{l c}
\hline
Jenis & Bilangan \\\hline
Nilam & 6\\
Berlian & 23\\
Emas & 56\\
Perak & 235\\
Gangsa & 324\\\hline
\end{tabular}
\end{table}

\begin{itemize}
  \item Etiam vitae pulvinar metus, sed fringilla orci
  \item Duis dapibus dolor risus, non ultrices enim porta sit amet.
  \item Ut eu libero augue.
\end{itemize}