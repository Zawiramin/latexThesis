%% Contoh tesis GayaUKM dalam Bahasa Melayu
\documentclass[bahasam,nohyphen]{GayaUKM}

\usepackage{graphicx}

\title{Pemodelan Persamaan Berjujukan Faktor Terpendam Non-Kognitif, Kadar Literasi Sains Pelajar, dan Pencapaian Sains Pelajar Malaysia}
\author{Muhammad Zawir Amin Bin Ahmad Farid}
\authorid{P90498}
\faculty{Fakulti Sains dan Teknologi}
\submissiondate{2 Oktober 2021}
\submissionyear{2021}
\degreetype{Doktor Falsafah}
\campus{Bangi}

%% If you find the boxes around hyperlinks distracting
\hypersetup{colorlinks,allcolors=black}

\begin{document}

\maketitle

\frontmatter
\declaration

% penghargaan dari penghargaan.tex
\input{penghargaan}

% abstrak dlm Bahasa Melayu dari abstrak-ms.tex
\input{abstrak-ms}

% abstrak dlm Bahasa Inggeris dari abstract-en.tex
\input{abstract-en}


\tableofcontents\clearpage
\listoffigures\clearpage
\listoftables\clearpage

% Senari simbol dll boleh disediakan seperti
% dalam senaraisimbol.tex
\input{senaraisimbol}


\mainmatter
% Setiap satu bab dari fail berasingan
\chapter{Pendahuluan}

\section{Pengenalan}
Bab ini menghuraikan pengenalan kepada kajian secara komprehensif. Perbincangan meliputi latar belakang kajian yang menghuraikan isu data raya dalam konteks pendidikan, peramalan pencapaian akademik, dan keutamaan penggunaan faktor non-kognitif dalam model peramalan terhadap murid yang menduduki ujian pentaksiran antarabangsa, PISA. Pernyataan masalah pula merangkumi permasalahan dalam membina sebuah model peramalan pencapaian pendidikan murid dan faktor non-kognitif sebagai pemboleh ubah utama dalam pembinaan model yang dipercayai antara faktor terpenting lain yang perlu diberi perhatian dalam penentuan pencapaian dan hala tuju masa hadapan murid. Bab ini juga menggariskan tujuan, objektif, persoalan, dan hipotesis kajian serta kepentingan dan batasan kajian ini juga turut diperincikan. 

\section{Latar Belakang Kajian}

\section{Pernyataan Masalah}
\subsection{Hipotesis Kajian}
\subsection{Persoalan Kajian}
Berikut adalah persoalan-persoalan kajian yang akan diterokai:
\begin{enumerate}
  \item Apakah wawasan yang diperolehi melalui teknik perlombongan data pembelajaran berpenyelia terhadap data PISA Malaysia?
  \item Sejauh manakah penggunaan model peramalan (pembelajaran berpenyelia) dalam data pendidikan peringkat sekolah?
  \item Apakah kesesuaian penggunaan teknik pembelajaran mesin pembelajaran berpenyelia dalam membina model peramalan prestasi murid di Malaysia?
  \item Apakah kelebihan penggunaan teknik pembelajaran mesin (pembelajaran berpenyelia) ke atas teknik-teknik statistik tradisional dalam meramalkan pencapaian murid sekolah?
  \item Apakah cara untuk merangka model pepohon keputusan (pembelajaran berpenyelia) bagi sebuah model peramalan prestasi murid di Malaysia?
  \item Adakah model lain pembelajaran berpenyelia mampu memperbaiki ketepatan model peramalan pepohon keputusan?
  \item Domain non-kognitif kurang mendapat perhatian. Apakah dapatan ramalan menggunakan teknik pembelajaran mesin pembelajaran berpenyelia terhadap faktor non-kognitif sains pelajar dalam ujian pentaksiran PISA 2015?
\end{enumerate}

\section{Tujuan Kajian}

\section{Objektif Kajian}

\section{Skop Kajian}

\section{Metodologi Kajian}

\section{Sumbangan Kajian}

\section{Organisasi Tesis}

\begin{table}[hbt!]\centering
\caption{Bilangan permata}
\begin{tabular}{l c}
\hline
Jenis & Bilangan \\\hline
Nilam & 6\\
Berlian & 23\\
Emas & 56\\
Perak & 235\\
Gangsa & 324\\\hline
\end{tabular}
\end{table}

\begin{itemize}
  \item Etiam vitae pulvinar metus, sed fringilla orci
  \item Duis dapibus dolor risus, non ultrices enim porta sit amet.
  \item Ut eu libero augue.
\end{itemize}
\input{bab-2}
\chapter{Metodologi Kajian}

\section{Pengenalan}
Bab ini membincangkan metodologi yang digunakan bagi menjawab persoalan kajian seperti di Bab 1. Perbincangan merangkumi kesemua aspek metodologi kajian yang dimulai dengan reka bentuk kajian, persediaan  dan pemilihan data, pengendalian data hilang,  pemilihan pemboleh ubah kajian, pembangunan model-model pembelajaran berpenyelia, pengujian dan penilaian model, serta analisis univariat dan bivariat terhadap data. Penerangan bagi setiap aspek dalam analisis data diberikan selepas ini. 

\section{Reka Bentuk Kajian}
Kajian ini menggunakan kaedah kuantitatif dan teknik perlombongan data sebagai tulang belakang kajian ini. Oleh kerana kajian ini menggunakan data PISA 2015, maka pengumpulan data kuantitatif telah dilaksanakan sepenuhnya oleh OECD. Analisis mendalam terhadap data PISA 2015 dilaksanakan oleh pengkaji bagi menjawab persoalan kajian 1 hingga 7, seperti mana yang telah dijelaskan dalam bahagian 1.5. Secara spesifik, kajian ini melibatkan analisis mendalam terhadap model pembelajaran berpenyelia, pepohon keputusan, terhadap faktor-faktor non-kognitif daripada Soal Selidik Konteks bagi PISA 2015

Kesan mengasingkan pemboleh ubah tidak signifikan daripada hasil keputusan awal peramalan model, terhadap tahap ketepatan model juga turut dikaji. Kesan menambah faktor lain ke dalam model peramalan juga diberi tumpuan. Bagi tujuan ini, pengkaji membina model-model pembelajaran berpenyelia daripada pembelajaran mesin dan perbandingan pencapaian setiap model dilaporkan.

Di samping itu, bagi memahami dengan lebih lanjut [tajuk]

Sungguhpun kajian ini seakan menyamai reka bentuk

\subsection{Sumber Data}
Kajian ini menggunakan data PISA 2015. Pemilihan PISA 2015 sebagai sumber data bagi tujuan kajian adalah berdasarkan pencapaian-pencapaian kitaran PISA yang lalu dalam menjalankan ujian pentaksiran secara global terhadap tahap literasi murid berusia 15 tahun. Penyertaan sekolah-sekolah Malaysia dalam ujian pentaksiran antarabangsa ini juga merupakan antara sebab pengkaji memilih data PISA ini bagi mengkaji pencapaian murid terhadap domain yang diuji pada kitaran PISA 2015 ini. 

Pengumpulan data yang telah dilaksanakan sepenuhnya oleh OECD seperti yang telah pengkaji nyatakan pada 3.2, menjadikan sumber data kajian ini sebagai sumber data sekunder. Penggunaan data sekunder sebagai sumber utama data kajian adalah kerana ketersediaan data tersebut. Kelebihan menggunakan data daripada sumber data sekunder dapat disimpulkan bahawa ketersediaan data sekunder ini banyak membantu dalam penjimatan masa, tenaga, dan perbelanjaan, sebagaimana yang dinyatakan dalam literatur \cite{Ayob2018, Johnston2014, Panchenko2020}. Kewujudan data yang telah tersedia dan berkualiti tinggi seperti data PISA telah membolehkan pelbagai kajian lain dilaksanakan bagi menjawab persoalan kajian yang berlainan dengan dapatan asal kajian data sekunder tersebut. Menurut \citet{Johnston2014} seperti dinyatakan dalam kajian \citet{Ayob2018}, secara umumnya, pengujian kerangka teori dan hipotesis lain boleh dijalankan ke atas data sekunder berskala besar yang sama. 

Walau bagaimanapun, seperti mana yang telah dinyatakan oleh \citet{Ayob2018} bahawa penggunaan data sekunder ini perlu dilaksanakan dengan berhati-hati

\subsection{Sampel Murid}

\section{Persediaan Data}

\section{Pembangunan Model}

\section{Pengujian dan Penilaian Model}



% rujukan tersenarai dlm refs.bib
\bibliography{library}

\appendix
% Setiap satu bab apendiks dari fail berasingan
\input{ap-huraian}
\input{ap-perisian}
\end{document}
