\chapter{Metodologi Kajian}

\section{Pengenalan}
Bab ini membincangkan metodologi yang digunakan bagi menjawab persoalan kajian seperti di Bab 1. Perbincangan merangkumi kesemua aspek metodologi kajian yang dimulai dengan reka bentuk kajian, persediaan  dan pemilihan data, pengendalian data hilang,  pemilihan pemboleh ubah kajian, pembangunan model-model pembelajaran berpenyelia, pengujian dan penilaian model, serta analisis univariat dan bivariat terhadap data. Penerangan bagi setiap aspek dalam analisis data diberikan selepas ini. 

\section{Reka Bentuk Kajian}
Kajian ini menggunakan kaedah kuantitatif dan teknik perlombongan data sebagai tulang belakang kajian ini. Oleh kerana kajian ini menggunakan data PISA 2015, maka pengumpulan data kuantitatif telah dilaksanakan sepenuhnya oleh OECD. Analisis mendalam terhadap data PISA 2015 dilaksanakan oleh pengkaji bagi menjawab persoalan kajian 1 hingga 7, seperti mana yang telah dijelaskan dalam bahagian 1.5. Secara spesifik, kajian ini melibatkan analisis mendalam terhadap model pembelajaran berpenyelia, pepohon keputusan, terhadap faktor-faktor non-kognitif daripada Soal Selidik Konteks bagi PISA 2015

Kesan mengasingkan pemboleh ubah tidak signifikan daripada hasil keputusan awal peramalan model, terhadap tahap ketepatan model juga turut dikaji. Kesan menambah faktor lain ke dalam model peramalan juga diberi tumpuan. Bagi tujuan ini, pengkaji membina model-model pembelajaran berpenyelia daripada pembelajaran mesin dan perbandingan pencapaian setiap model dilaporkan.

Di samping itu, bagi memahami dengan lebih lanjut [tajuk]

Sungguhpun kajian ini seakan menyamai reka bentuk

\subsection{Sumber Data}
Kajian ini menggunakan data PISA 2015. Pemilihan PISA 2015 sebagai sumber data bagi tujuan kajian adalah berdasarkan pencapaian-pencapaian kitaran PISA yang lalu dalam menjalankan ujian pentaksiran secara global terhadap tahap literasi murid berusia 15 tahun. Penyertaan sekolah-sekolah Malaysia dalam ujian pentaksiran antarabangsa ini juga merupakan antara sebab pengkaji memilih data PISA ini bagi mengkaji pencapaian murid terhadap domain yang diuji pada kitaran PISA 2015 ini. 

Pengumpulan data yang telah dilaksanakan sepenuhnya oleh OECD seperti yang telah pengkaji nyatakan pada 3.2, menjadikan sumber data kajian ini sebagai sumber data sekunder. Penggunaan data sekunder sebagai sumber utama data kajian adalah kerana ketersediaan data tersebut. Kelebihan menggunakan data daripada sumber data sekunder dapat disimpulkan bahawa ketersediaan data sekunder ini banyak membantu dalam penjimatan masa, tenaga, dan perbelanjaan, sebagaimana yang dinyatakan dalam literatur \cite{Ayob2018, Johnston2014, Panchenko2020}. Kewujudan data yang telah tersedia dan berkualiti tinggi seperti data PISA telah membolehkan pelbagai kajian lain dilaksanakan bagi menjawab persoalan kajian yang berlainan dengan dapatan asal kajian data sekunder tersebut. Menurut \citet{Johnston2014} seperti dinyatakan dalam kajian \citet{Ayob2018}, secara umumnya, pengujian kerangka teori dan hipotesis lain boleh dijalankan ke atas data sekunder berskala besar yang sama. 

Walau bagaimanapun, seperti mana yang telah dinyatakan oleh \citet{Ayob2018} bahawa penggunaan data sekunder ini perlu dilaksanakan dengan berhati-hati

\subsection{Sampel Murid}

\section{Persediaan Data}

\section{Pembangunan Model}

\section{Pengujian dan Penilaian Model}
